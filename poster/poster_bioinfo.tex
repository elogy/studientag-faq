	\begin{LARGE}
				Bioinformatik
			\end{LARGE}
			\begin{exampleblock}{Was ist der Studiengang?}
				Grob gesagt die Schnittstelle zwischen dem Chemiker im Labor und der Datenverarbeitung am Rechner. Mögliche Schwerpunkte gehen in Richtung autoamtisierte Verarbeitung von DNA-Daten, Drug Design, Krebsforschung etc.
				Das Studium beinhaltet neben der klassischen Informatik Inhalte aus Molekularbiologie, Neurobiologie, Biochemie und Chemie. Ein Schwerpunktfach gibt es nicht.
			\end{exampleblock}
		
			\begin{block}{Welcher Teil macht wie viel im Studium aus?}
				\begin{figure}[h!]
					\caption{Verteilung der Themenbereiche über das komplette Studium}
				\end{figure}
			\end{block}
		
		\begin{block}{Was macht man in welchem Semester?}
			\begin{figure}[h!]
				\caption{Vorschlag für den Studienverlauf}
			\end{figure}
		Dieser Verlauf ist allerdings nur ein Vorschlag und kein bindender Studienplan.
		\end{block}